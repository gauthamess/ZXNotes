	\documentclass[11pt, oneside,reqno]{amsart}   	% use "amsart" instead of "article" for AMSLaTeX format
\usepackage{geometry}                		% See geometry.p\mathrm{d}F to learn the layout options. There are lots.
\geometry{letterpaper}                   		% ... or a4paper or a5paper or ... 
%\geometry{landscape}                		% Activate for rotated page geometry
%\usepackage[parfill]{parskip}    		% Activate to begin paragraphs with an empty line rather than an indent
\usepackage{graphicx}				% Use p\mathrm{d}F, png, jpg, or eps§ with p\mathrm{d}Flatex; use eps in DVI mode

\geometry{
	a4paper,
	total={170mm,207mm},
	left= 20mm,
	top=25mm,
	bottom = 30 mm,
}% TeX will automatically convert eps --> p\mathrm{d}F in p\mathrm{d}Flatex		

\usepackage{amsthm}
\usepackage{amsmath}
\usepackage{amssymb}
\usepackage{cancel}
\usepackage{hyperref}
\usepackage{enumitem}
\usepackage{etoolbox}

\usepackage{subcaption}
\usepackage{bbm}
\usepackage{mathtools}
\usepackage{xcolor}  
\usepackage{tikz}
\usepackage{etoolbox}
\usepackage[utf8]{inputenc}
\usepackage{parskip}
\usetikzlibrary{arrows.meta, positioning}
\usepackage{array}
\def\dbar{{\mathchar'26\mkern-12mu d}}

\usepackage{faktor}
\usepackage{zx-calculus}


\setcounter{secnumdepth}{3}
\setcounter{tocdepth}{3} 
\makeatletter
\def\blfootnote{\xdef\@thefnmark{}\@footnotetext}
\makeatother
\newcommand{\bd}{\textbf}
\newcommand{\bb}{\mathbb}
\newcommand{\mf}{\mathfrak}
\newcommand{~}{\noindent}
\newcommand{\Pos}{\text{Pos}}
\newcommand{\R}{\Rightarrow}
\newcommand{\h}{\hat{}}
\newcommand{\ah}{\hat{a}}
\newcommand{\mh}{\mathcal{H}}
\newcommand{\ka}{\ket{\alpha}}
\newcommand{\ba}{\bra{\alpha}}
\newcommand{\m}{\mathcal}
\newcommand\Tau{\mathcal{T}}
\newcommand{\im}{\mathrm{im}}
\newcommand{\norm}[1]{\left\lVert#1\right\rVert}
\newcommand{\bra}[1]{\langle#1|}
\newcommand{\ket}[1]{|#1\rangle}
\newcommand{\der}[2]{\frac{\mathrm{d}#1}{\mathrm{d}#2}} % Ordinary derivative
\newcommand{\pder}[2]{\frac{\partial #1}{\partial #2}} % Partial derivative
\newcommand{\nder}[3]{\frac{\mathrm{d}^{#1}#2}{\mathrm{d}#3^{#1}}} % Higher-order ordinary derivative
\newcommand{\npder}[3]{\frac{\partial^{#1}#2}{\partial #3^{#1}}} % Higher-order partial derivative
\newcommand{\cc}[1]{\mathcal{C}^#1}
\newcommand{\diff}{\mathrm{d}}

\usepackage{pdfpages}


\DeclareMathOperator{\Hom}{Hom}
\DeclareMathOperator{\vecc}{vec}



\newcommand{\pcv}[3]{\left( \pder{#1}{#2} \right)_{#3}}

\newcommand{\I}{\implies}
\newcommand{\quo}{\ensuremath{\mathrel{\mathsf{div}}}}
\newcommand{\rel}{\ensuremath{\mathbb{R}}}
\newcommand{\nat}{\ensuremath{\mathbb{N}}}
\newcommand{\ing}{\ensuremath{\mathbb{Z}}}
\newcommand{\clx}{\ensuremath{\mathbb{C}}}
\newcommand{\kel}{\ensuremath{\mathbb{K}}}
\newcommand{\rand}{\text{ and }}
\newcommand{\qedd}{&&\text{\qed}}
\newcommand{\et}[1]{&&\text{(#1)}\\}
\newcommand{\pbc}{&\text{\emph{(Proof by Contradiction)}}}

\newtheorem{theorem}{Theorem}
\newtheorem{conclusion}[theorem]{Conclusion}
\newtheorem{corollary}[theorem]{Corollary}
\newtheorem{definition}[theorem]{Definition}
\newtheorem{proposition}[theorem]{Proposition}
\newtheorem{lemma}[theorem]{Lemma}
\newtheorem{remark}[theorem]{Remark}

\newcommand{\mycomment}[1]{}
\def\centerarc[#1](#2)(#3:#4:#5)% Syntax: [draw options] (center) (initial angle:final angle:radius)
{ \draw[#1] ($(#2)+({#5*cos(#3)},{#5*sin(#3)})$) arc (#3:#4:#5); }



\setlength{\parindent}{0pt}


%SetFonts

%SetFonts


\title{
	ZX Notes }
\date{Y}


\begin{document}
	\maketitle
	
	\includepdf[pages=-]{ideas-gautham.pdf}
	
	
\section{ZX Introduction}

\section{Measurements}

\subsection{Pauli Group}

\begin{definition}
	\bd{Pauli strings. } \[ P_n = \{ \bigotimes_{i=1}^n A_i | A_i \in P \} \] where, $P$ is the Pauli group, \[P := \alpha\{I,X,Y,Z\}, \alpha \in \{\pm 1, \pm i\}  \]
	and $I,X,Y,Z$ have matrix representation,
	\[ I = \begin{pmatrix}
		1 & 0 \\ 0 & 1
	\end{pmatrix},  X = \begin{pmatrix}
	0 & 1 \\ 1 & 0
	\end{pmatrix}, Y = \begin{pmatrix}
	0 & -i \\ i & 0
	\end{pmatrix},  Z = \begin{pmatrix}
 1 & 0 \\ 0 & -1
	\end{pmatrix}  \]
\end{definition}

Note that this is an \emph{irreducible representation}. We can check that $X$ and $Z$ do not share any eigenvectors (and so any invariant 1dim subspaces).
		
$\mathbb{Q}_8$ is a subgroup of $P$. $P$ is the smallest subgroup (of $U(2)$) generated by $\langle X,Y,Z \rangle$.

Quotienting $P$ by the center yields the Klein four-group (TODO with other properties and lie algebra lie group).

Note that, 
\[ SU(2) \cap P_1 = \pm1 \{ I\} \cup \pm i \{ X,Y,Z \} \leq P_1\]
This subgroup is isomorphic to $\mathbb{Q}_8 = \{ \pm 1, \pm i, \pm j, \pm k  \}$ via the identification.
\[ i \to iX, j \to iY, k \to iZ\]
		
\subsection{Properties}

\subsection{Measurements}
In an MBQC we need to specify measurement operations for each non-output. We do this by assigning \emph{measurement planes} for each qubit,

Any general measurement for a single qubit is specified by an axis on the Bloch sphere. Convention is to restrict these axes to a plane of the Bloch sphere -- $XY$, $YZ$, or $XZ$. The axes selects two states -- $\ket{\eta}$ and $\ket{\eta'}$ which is diametrically opposite on the sphere, then we form,
\[ \Pi = \ket{\eta} \bra{\eta} + \ket{\eta'}\bra{\eta'}\]
and make a projective measurement.

We can write these states and axes explicitly. The Bloch sphere is parametrised as $(\theta, \varphi)$ -- azimuthal and polar. We choose our axes by fixing one of the angles, \[\theta = \pi/2 \ (XY), \varphi = 0 \ (YZ) \text{ or } \varphi = \pi/2 \ (XZ)\]
	\begin{align*}
		\ket{+_{XY}} = \frac{1}{\sqrt{2}} (\ket{0} + e^{i\alpha} \ket{1})&&\ket{-_{XY}} =\frac{1}{\sqrt{2}} (\ket{0} - e^{i\alpha} \ket{1})\\
		\ket{+_{XZ}} = \cos \frac{\alpha}{2} \ket{0} + \sin \frac{\alpha}{2} \ket{1} &&\ket{-_{XZ}}  = \sin \frac{\alpha}{2} \ket{0}  -  \cos \frac{\alpha}{2} \ket{1}\\
		\ket{+_{YZ}} = \cos \frac{\alpha}{2} \ket{0} + i \sin \frac{\alpha}{2}\ket{1} &&\ket{-_{YZ} } =  \sin \frac{\alpha}{2} \ket{0} - i \cos \frac{\alpha}{2}\ket{1}
			\end{align*}
			Note that any point (pure state) on the boundary of the Bloch sphere can be written as, \[(\theta, \varphi) \mapsto \cos \frac{\varphi}{2} \ket{0} + e^{i \theta} \sin \frac{\varphi}{2} \ket{1}\]
			The measurement axis coincide with $X,Y$ or $Z$ corresponds to $\alpha = a \pi$ with $a \in \{0,1\}$ \footnote{more exactly $\alpha = \frac{2 a}{\pi} \mod 4, a \in \{0,1,2,3\}$ to pick out both axes in each plane.}
			
			After picking an axis, we construct our measurement as a projector,
			\[ \Pi_{AB,\alpha} = \ket{+_{AB}(\alpha)} \bra{+_{AB}(\alpha)} -  \ket{-_{AB}(\alpha)} \bra{-_{AB}(\alpha)}\]
			Typically in MBQC, our ``desired outcome" is the +1 eigenvalue collapse to the $\ket{+}$. This is usually denoted as outcome 0, and the undesired $\ket{-}$ collapse as outcome 1.
			
			Depending on the plane of measurement, we can apply a pauli $X,Y$ or $Z$ to correct it. $Z$ changes the relative phase, $X$ swaps $0 \rand 1$ and $ZX = Y$ does both.
			\[ Z \ket{-_{XY}} = \ket{+_{XY}}, \ Y \ket{-_{XZ}} = \ket{+_{XZ}} \rand X \ket{-_{YZ}} = \ket{+_{YZ}}\]			
\subsection{Stabilisers}

\begin{definition}
	The \bd{Clifford group} is the \emph{normaliser} of the Pauli group $P_n$ in $U(2^n)$,
	\[ C_n := \{ g \in SU(2^n) | g P_n g^{-1} = P_n\}\]
	A \bd{Clifford gate} is an element of $C_n$.
	\end{definition}
	
	\emph{fact:} the Clifford group on $n$-qubits is generated by Hadamard, Phase ($i$) and CNOT gates. 
	
	\begin{theorem}
		\bd{(Gottesmann-Knill). }Any circuit involving only Clifford gates and Pauli measurements is (polynomial time) easily simulatable.  
	\end{theorem}
	So, we want to try to minimize the number of non-Clifford gates in our circuit.
	
	\emph{fact: }$C_n$ is not a universal gate-set. The $T$ gate cannot be finitely generated.
	\[ T = \begin{pmatrix}
		1 & 0 \\ 0 & e^{i \pi/4}
	\end{pmatrix}\]
	Note that the Clifford group is not finite because if $g$ normalises $P_n$ then so does $e^{i \phi} g$. We can disregard global phases and just consider, $C_n / U(1)$.
				
\section{MBQC }

\section{References }
1. Pauli and Stabilisers -- Watrous TQI\\
2. MBQC -- \\
3. PDDAG --


\newpage
\subsection{Stabilizer–Based Feed‐Forward Corrections in MBQC}

Let $G=(V,E)$ be our open graph with inputs $I\subset V$ and outputs $O\subset V$, and let $\ket{G}$ be the associated graph‐state on $V$:

\[
\mathcal S_G \;=\;\bigl\langle\,K_u = X_u\!\!\prod_{v\in N(u)}Z_v
: u\in V\bigr\rangle,
\qquad
K_u\ket{G} = +\,\ket{G}
\quad\forall\,u.
\]

We process measurements in an order consistent with a flow $f$ on $(G,I,O)$.
\medskip
\noindent\textbf{Initialization.}
\begin{itemize}
	\item For each non‐output qubit $u\in V\setminus O$, choose:
	\begin{itemize}
		\item A measurement plane $\lambda(u)\in\{XY,XZ,YZ\}$.
		\item A \emph{base angle} $\phi_u\in[0,2\pi)$.
	\end{itemize}
	\item Initialize two counters on every qubit $v\in V$,
	\[
	c_X(v) \;=\; 0,
	\qquad
	c_Z(v) \;=\; 0.
	\]
	These record how many $\pi$–shifts have been accumulated for the $X$– and $Z$–components of $v$’s basis.
\end{itemize}

\medskip
\noindent\textbf{Online Loop.}  Iterate over $u\in V\setminus O$ in flow order:

\begin{enumerate}
	\item \emph{Compute the adaptive angle.}  If $\lambda(u)=XY$, set
	\[
	\alpha_u \;=\;\phi_u \;+\;\pi\;c_X(u).
	\]
	If $\lambda(u)=XZ$, use $\pi\,c_Z(u)$ in place of $\pi\,c_X(u)$, and similarly for $YZ$ by combining both.
	
	\item \emph{Measure qubit $u$.}  Project onto
	\[
	\Pi_u^{(s_u)}
	= \frac{1 + (-1)^{s_u}\,P_u(\alpha_u)}{2},
	\quad
	P_u(\alpha_u)
	= \cos\alpha_u\,X_u + \sin\alpha_u\,Y_u
	\]
	let $s_u\in\{0,1\}$ be the outcome bit (0 for “+”, 1 for “–”).
	
	\item \emph{Syndrome‐based feed‐forward.}  For each \emph{future} qubit $v\succ u$:
	\[
	c_X(v)\;\longmapsto\;c_X(v)+s_u
	\quad\text{if }[P_v,\,P_u(\alpha_u)]\neq0\text{ in the $X$–component},
	\]
	and
	\[
	c_Z(v)\;\longmapsto\;c_Z(v)+s_u
	\quad\text{if }[P_v,\,P_u(\alpha_u)]\neq0\text{ in the $Z$–component}.
	\]
	Equivalently, $v$ lies in the \emph{anticommuting neighbourhood} of $u$ under the stabilizers $K_v$.
\end{enumerate}

\medskip
\noindent\textbf{Final Correction on Outputs.}
\[
\text{On each }o\in O,\;
\text{apply }X^{\,c_X(o)}Z^{\,c_Z(o)}
\quad\text{(or flip the classical output bits accordingly).}
\]

\bigskip
\noindent\emph{Why this works:}  
\begin{itemize}
	\item Each measurement at $u$ “flips” those stabilizers $K_v$ which anticommute with $P_u(\alpha_u)$, exactly encoding the syndrome bit $s_u$.
	\item Adding $\pi\,s_u$ to the angle of each affected future $v$ cancels the corresponding $(-1)^{s_u}$ byproduct.
	\item Because the flow order guarantees causality ($u$ only affects $v\succ u$), all byproducts are absorbed before reaching the outputs, yielding a deterministic overall map.
\end{itemize}

\section{Pauli Flow}


\section{PDDAG}




\end{document}
