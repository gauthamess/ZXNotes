   	\documentclass[11pt, oneside,reqno]{amsart}   	% use "amsart" instead of "article" for AMSLaTeX format
\usepackage{geometry}                		% See geometry.p\mathrm{d}F to learn the layout options. There are lots.
\geometry{letterpaper}                   		% ... or a4paper or a5paper or ... 
%\geometry{landscape}                		% Activate for rotated page geometry
%\usepackage[parfill]{parskip}    		% Activate to begin paragraphs with an empty line rather than an indent
\usepackage{graphicx}				% Use p\mathrm{d}F, png, jpg, or eps§ with p\mathrm{d}Flatex; use eps in DVI mode

\geometry{
	a4paper,
	total={170mm,207mm},
	left= 20mm,
	top=25mm,
	bottom = 30 mm,
}% TeX will automatically convert eps --> p\mathrm{d}F in p\mathrm{d}Flatex		

\usepackage{amsthm}
\usepackage{amsmath}
\usepackage{amssymb}
\usepackage{cancel}
\usepackage{hyperref}
\usepackage{enumitem}
\usepackage{etoolbox}

\usepackage{subcaption}
\usepackage{bbm}
\usepackage{mathtools}
\usepackage{xcolor}  
\usepackage{tikz}
\usepackage{etoolbox}
\usepackage[utf8]{inputenc}
\usepackage{parskip}
\usetikzlibrary{arrows.meta, positioning}
\usepackage{array}
\def\dbar{{\mathchar'26\mkern-12mu d}}

\usepackage{faktor}
\usepackage{zx-calculus}


\setcounter{secnumdepth}{3}
\setcounter{tocdepth}{3} 
\makeatletter
\def\blfootnote{\xdef\@thefnmark{}\@footnotetext}
\makeatother
\newcommand{\bd}{\textbf}
\newcommand{\bb}{\mathbb}
\newcommand{\mf}{\mathfrak}
\newcommand{~}{\noindent}
\newcommand{\Pos}{\text{Pos}}
\newcommand{\R}{\Rightarrow}
\newcommand{\h}{\hat{}}
\newcommand{\ah}{\hat{a}}
\newcommand{\mh}{\mathcal{H}}
\newcommand{\ka}{\ket{\alpha}}
\newcommand{\ba}{\bra{\alpha}}
\newcommand{\m}{\mathcal}
\newcommand\Tau{\mathcal{T}}
\newcommand{\im}{\mathrm{im}}
\newcommand{\norm}[1]{\left\lVert#1\right\rVert}
\newcommand{\bra}[1]{\langle#1|}
\newcommand{\ket}[1]{|#1\rangle}
\newcommand{\der}[2]{\frac{\mathrm{d}#1}{\mathrm{d}#2}} % Ordinary derivative
\newcommand{\pder}[2]{\frac{\partial #1}{\partial #2}} % Partial derivative
\newcommand{\nder}[3]{\frac{\mathrm{d}^{#1}#2}{\mathrm{d}#3^{#1}}} % Higher-order ordinary derivative
\newcommand{\npder}[3]{\frac{\partial^{#1}#2}{\partial #3^{#1}}} % Higher-order partial derivative
\newcommand{\cc}[1]{\mathcal{C}^#1}
\newcommand{\diff}{\mathrm{d}}

\usepackage{pdfpages}


\DeclareMathOperator{\Hom}{Hom}
\DeclareMathOperator{\vecc}{vec}



\newcommand{\pcv}[3]{\left( \pder{#1}{#2} \right)_{#3}}

\newcommand{\I}{\implies}
\newcommand{\quo}{\ensuremath{\mathrel{\mathsf{div}}}}
\newcommand{\rel}{\ensuremath{\mathbb{R}}}
\newcommand{\nat}{\ensuremath{\mathbb{N}}}
\newcommand{\ing}{\ensuremath{\mathbb{Z}}}
\newcommand{\clx}{\ensuremath{\mathbb{C}}}
\newcommand{\kel}{\ensuremath{\mathbb{K}}}
\newcommand{\rand}{\text{ and }}
\newcommand{\qedd}{&&\text{\qed}}
\newcommand{\et}[1]{&&\text{(#1)}\\}
\newcommand{\pbc}{&\text{\emph{(Proof by Contradiction)}}}

\newtheorem{theorem}{Theorem}
\newtheorem{conclusion}[theorem]{Conclusion}
\newtheorem{corollary}[theorem]{Corollary}
\newtheorem{definition}[theorem]{Definition}
\newtheorem{proposition}[theorem]{Proposition}
\newtheorem{lemma}[theorem]{Lemma}
\newtheorem{remark}[theorem]{Remark}

\newcommand{\mycomment}[1]{}
\def\centerarc[#1](#2)(#3:#4:#5)% Syntax: [draw options] (center) (initial angle:final angle:radius)
{ \draw[#1] ($(#2)+({#5*cos(#3)},{#5*sin(#3)})$) arc (#3:#4:#5); }



\setlength{\parindent}{0pt}


%SetFonts

%SetFonts


\title{
	ZX Notes }
\date{Y}


\begin{document}
	\maketitle
	
	\includepdf[pages=-]{ideas-gautham.pdf}
	\section{ZX Introduction}
	
	\section{Measurements}
	
	\subsection{Pauli Group}
	
	\begin{definition}
		\bd{Pauli strings. } \begin{equation} P_n = \{ \bigotimes_{i=1}^n A_i | A_i \in P \} \end{equation} where, $P$ is the Pauli group, \begin{equation}P := \alpha\{I,X,Y,Z\}, \alpha \in \{\pm 1, \pm i\}  \end{equation}
		and $I,X,Y,Z$ have matrix representation,
		\begin{equation} I = \begin{pmatrix}
				1 & 0 \\ 0 & 1
			\end{pmatrix},  X = \begin{pmatrix}
				0 & 1 \\ 1 & 0
			\end{pmatrix}, Y = \begin{pmatrix}
				0 & -i \\ i & 0
			\end{pmatrix},  Z = \begin{pmatrix}
				1 & 0 \\ 0 & -1
		\end{pmatrix}  \end{equation}
	\end{definition}
	
	Note that this is an \emph{irreducible representation}. We can check that $X$ and $Z$ do not share any eigenvectors (and so any invariant 1dim subspaces).
	
	$\mathbb{Q}_8$ is a subgroup of $P$. $P$ is the smallest subgroup (of $U(2)$) generated by $\langle X,Y,Z \rangle$.
	
	Quotienting $P$ by the center yields the Klein four-group (TODO with other properties and lie algebra lie group).
	
	Note that, 
	\begin{equation} SU(2) \cap P_1 = \pm1 \{ I\} \cup \pm i \{ X,Y,Z \} \leq P_1\end{equation}
	This subgroup is isomorphic to $\mathbb{Q}_8 = \{ \pm 1, \pm i, \pm j, \pm k  \}$ via the identification.
	\begin{equation} i \to iX, j \to iY, k \to iZ\end{equation}
	
	\subsection{Properties}
	
	\subsection{Measurements} \label{meas}
	In an MBQC we need to specify measurement operations for each non-output. We do this by assigning \emph{measurement planes} for each qubit,
	
	Any general measurement for a single qubit is specified by an axis on the Bloch sphere. Convention is to restrict these axes to a plane of the Bloch sphere -- $XY$, $YZ$, or $XZ$. The axes selects two states -- $\ket{\eta}$ and $\ket{\eta'}$ which are diametrically opposite on the sphere, then we form,
	\begin{equation} \Pi = \ket{\eta} \bra{\eta} + \ket{\eta'}\bra{\eta'}\end{equation}
	and make a projective measurement.
	
	We can write these states and axes explicitly. The Bloch sphere is parametrised as $(\theta, \varphi)$ -- azimuthal and polar. We choose our axes by fixing one of the angles, \begin{equation}\theta = \pi/2 \ (XY), \varphi = 0 \ (YZ) \text{ or } \varphi = \pi/2 \ (XZ)\end{equation}
	\begin{align*}
		\ket{+_{XY}} = \frac{1}{\sqrt{2}} (\ket{0} + e^{i\alpha} \ket{1})&&\ket{-_{XY}} =\frac{1}{\sqrt{2}} (\ket{0} - e^{i\alpha} \ket{1})\\
		\ket{+_{XZ}} = \cos \frac{\alpha}{2} \ket{0} + \sin \frac{\alpha}{2} \ket{1} &&\ket{-_{XZ}}  = \sin \frac{\alpha}{2} \ket{0}  -  \cos \frac{\alpha}{2} \ket{1}\\
		\ket{+_{YZ}} = \cos \frac{\alpha}{2} \ket{0} + i \sin \frac{\alpha}{2}\ket{1} &&\ket{-_{YZ} } =  \sin \frac{\alpha}{2} \ket{0} - i \cos \frac{\alpha}{2}\ket{1}
	\end{align*}
	Note that any point (pure state) on the boundary of the Bloch sphere can be written as, \begin{equation}(\theta, \varphi) \mapsto \cos \frac{\varphi}{2} \ket{0} + e^{i \theta} \sin \frac{\varphi}{2} \ket{1}\end{equation}
	The measurement axis coincide with $X,Y$ or $Z$ corresponds to $\alpha = a \pi$ with $a \in \{0,1\}$ \footnote{more exactly $\alpha = \frac{2 a}{\pi} \mod 4, a \in \{0,1,2,3\}$ to pick out both axes in each plane.}
	
	After picking an axis, we construct our measurement as a projector,
	\begin{equation} \Pi_{\bullet \circ,\alpha} = \ket{+_{\bullet \circ}(\alpha)} \bra{+_{\bullet \circ}(\alpha)} -  \ket{-_{\bullet \circ}(\alpha)} \bra{-_{\bullet \circ}(\alpha)}\end{equation}
	
	It is useful to construct the measured Hermitian operators (with eigenbasis $\ket{+},\ket{-}$) corresponding to each planar measurement.
	\begin{align*}
		XY \mapsto \cos \alpha X + \sin \alpha Y =: M_{XY,\alpha} \\
		XZ \mapsto \sin \alpha X + \cos \alpha Z =: M_{XZ,\alpha}  \\
		YZ \mapsto \sin \alpha Y + \cos \alpha Z =: M_{YZ,\alpha} 
	\end{align*}
	Typically in MBQC, our ``desired outcome" is the +1 eigenvalue collapse to the $\ket{+}$. This is usually denoted as outcome 0, and the undesired $\ket{-}$ collapse as outcome 1.
	
	Depending on the plane of measurement, we can apply a pauli $X,Y$ or $Z$ to correct it. $Z$ changes the relative phase, $X$ swaps $0 \rand 1$ and $ZX = Y$ does both.
	\begin{equation} Z \ket{-_{XY}} = \ket{+_{XY}}, \ Y \ket{-_{XZ}} = \ket{+_{XZ}} \rand X \ket{-_{YZ}} = \ket{+_{YZ}}\end{equation}			
	\section{Stabilisers}
	The idea is to specify a state (uniquely) via the generators of its stabiliser subgroup. ``Errors" become changes to generators.
	\subsection{Clifford group}
	\begin{definition}
		The \bd{Clifford group} is the \emph{normaliser} of the Pauli group $P_n$ in $U(2^n)$,
		\begin{equation} C_n := \{ g \in SU(2^n) | g P_n g^{-1} = P_n\}\end{equation}
		A \bd{Clifford gate} is an element of $C_n$.
	\end{definition}
	
	\emph{fact:} the Clifford group on $n$-qubits is generated by Hadamard, Phase ($i$) and CNOT gates. 
	
	\begin{theorem}
		\bd{(Gottesmann-Knill). }Any circuit involving only initial state $\ket{0}^{\otimes n}$, Clifford gates and Pauli measurements is (polynomial time) easily simulatable.  
	\end{theorem}
	So, we want to try to minimize the number of non-Clifford gates in our circuit.
	
	\emph{fact: }$C_n$ is not a universal gate-set. The $T$ gate cannot be finitely generated.
	\begin{equation} T = \begin{pmatrix}
			1 & 0 \\ 0 & e^{i \pi/4}
	\end{pmatrix}\end{equation}
	Note that the Clifford group is not finite because if $g$ normalises $P_n$ then so does $e^{i \phi} g$. We can disregard global phases and just consider $C_n / U(1)$.
	\subsection{Stabilisers and States}
	\begin{definition}
		The \bd{subspace stabilised} by a subgroup $H \leq P_n$ is,
		\begin{equation}V_H := \{ \ket{\psi} \in (\clx^2)^{\otimes n} | h \ket{\psi} = \ket{\psi} \forall h \in H \}\end{equation}
	\end{definition}
	\begin{proposition}
		For all $g \in U(2^n)$ and $H \leq P_n$, $V_{gHg^{-1}} = g V_H$.
	\end{proposition}
	%			i.e, the stabiliser subgroup is fixed under conjugation by $U(2^n)$.map
	
	Note that if $- \mathbbm{1} \in H$, then $-\mathbbm{1} \ket{\psi} = \ket{\psi} \implies \ket{\psi} = 0$. This must be excluded for the stabiliser space to be non-trivial.
	\begin{proposition}
		If $H \leq P_n$ and $\dim(V_H) > 0$, then $H$ is abelian and $-\mathbbm{1} \notin H$.
	\end{proposition}
	\begin{proof}
		$g_1, g_2 \in P_n$ either commute or anticommute. If they anticommute then $g_1 g_2 \ket{\psi} = -g_2 g_1 \ket{\psi} \implies \ket{\psi} = 0$
	\end{proof}
	\begin{definition}
		A set $S \leq H$ is \bd{independent} if for all $g \in S$,
		\begin{equation} \langle S \setminus \{g\} \rangle \neq \langle S \rangle \end{equation}
	\end{definition} 
	And finally, the theorem below allows us to specify a unique state by specifying $n$ commuting Pauli strings. Let us denote $V_S$ to be the subspace stabilised by a generating set $\langle S \rangle$.
	\begin{theorem}
		If $S = \{g_1, \ldots, g_l\}$ is indepdent, pairwise commutative such that $- \mathbbm{1} \notin S$, then $\dim(V_S) = 2^{n-l}$.
	\end{theorem}
	
	There is a surjective group homomorphism, $r : P_n \to \mathbb{F}_2^{2n}$ with $\mathrm{ker}(r) = \{ \pm 1, \pm i \mathbbm{1}\}$
	via the map,
	\begin{equation} r(X_i) = e_i \rand r(Z_i) = e_{i+n}\end{equation}
	we can keep track of $X_i,Y_i$ via this map as a $2 \times 2n$ matrix called the check matrix.
	
	For example, the bell state $(\ket{00} + \ket{11})$ is the stabiliser state of, 
	\begin{equation} \langle X_1X_2,Z_1Z_2 \rangle\end{equation}
	with associated check matrix,
	\begin{equation} \begin{pmatrix}
			1 & 1 & 0 & 0 \\ 0 & 0 & 1 & 1
	\end{pmatrix}\end{equation}
	
	The stabiliser tableaux is an extension of the check matrix of the state.
	
	Since we can specify states via its stabiliser (group), we can also track measurements. First, note that for any $g \in P_n$, the projectors $P_+, P_-$ can be written as,
	\begin{equation} P_+ = \frac{1}{2}(I+g) \rand P_- = \frac{1}{2}(I-g) \end{equation}
	with probabiltiies,
	\begin{equation}
		p(+) = \frac{1}{2}(1 + \langle \psi | \tilde g | \psi \rangle ) \rand
		p(-) = \frac{1}{2}(1 - \langle \psi | \tilde g | \psi \rangle ) 
	\end{equation}
	
	We want to measure some observable $\tilde g \in P_n$. 
	
	It is easy to check that any $g \in P_n$ either commutes or anticommutes with $\tilde g$. Consider a stabiliser $V_S$ generated by $S = \langle g_1, \ldots, g_n \rangle$. There are two cases,
	
	(i) $\tilde g$ commutes with all stabilisers.  
	
	Then, for any $g$, $g (\tilde g \ket{\psi}) = \tilde g \ket{\psi}$. So, $\tilde g \ket{\psi}$ is a common $+1$ eigenvector of all stabilisers. Since $V_S$ is one-dimensional, $\tilde g \ket{\psi} \propto \ket{\psi}$, but because it has eigenvalues $\pm 1$, $\tilde g\ket{\psi} = \pm \ket{\psi}$. So, one of $\pm \tilde g$ is a stabiliser.
	
	The outcome is deterministic $+$ or $-$ if $+ \tilde g \in S$ or $- \tilde g \in S$. We do not need to update the stabilisers,
	\begin{equation}
		S = \langle g_1, \ldots, g_n \rangle
	\end{equation}
	
	(ii) $\tilde g$ anticommutes with some stabilisers.
	
	wlog, we can assume that $\tilde g$ anticommutes with $g_1$ and commutes with all other $g_i$ by picking $g_1$ such that $\{\tilde g, g_1\} =0$ and for other $i$, $g_i \mapsto g_1 g_i =: g'_i$,
	\begin{equation} \label{stab_update}
		g g'_i = g(g_1 g_i) = - g_1 (g g_i) = -g_1 (-g_i g) = g'_i g 
	\end{equation}
	note that this is just a change of choice of generators and does not change $S$.
	
	In this case, the measurement is \emph{not} deterministic because,
	\begin{equation}
		p(+) = \frac{1}{2}(1 + \bra{\psi} \tilde g \ket{\psi}) = \frac{1}{2}(1 + \bra{\psi} \tilde g g_1\ket{\psi}) = \frac{1}{2}(1 - \bra{\psi} \tilde g \ket{\psi}) = p(-) = \frac{1}{2}
	\end{equation}
	In this case, depending on the measurement outcome we update the stabiliser (after measurement) to,
	\begin{equation}
		S = \langle \pm \tilde g, g_2, \ldots, g_n \rangle
	\end{equation}
	
	The GHZ state for example cannot be described by a stabiliser because $p(Z_1 = +) = \frac{1}{3}$.
	
	Note that all of this assumes that we are making a \emph{pauli measurement}. In an MBQC, before we make the $X_u$ local measurement, we must update the tableaux to the above form -- anticommuting with one (neighbour) vertex and commuting with all others as in \ref{stab_update}.
	
	This update for a $X$-measurement is the exact same as local complementation on the measured vertex (prove later).
	


	\section{MBQC }
	
	MBQC leverages entanglement (teleportation) instead of unitary gates to create a one-way model of quantum computation.
	
	\begin{definition}
		A \bd{measurement pattern} consists of an $n$-qubit register $V$ with distinguished sets $I,O \subseteq V$ of inputs and outputs. Additionally there is a sequence of the following operations:
		
		i. Preperations -- intialising all qubits $i \notin I$ in the $\ket{+}$ state.
		
		ii. Entangling -- applying a $CZ_{ij}$ to qubit $i$ as control and qubit $j$ as target for (chosen) pairs $(i,j) \in V \times V$.
		
		iii. Destructive measurements --  which project qubits $i \notin O$ onto orthonormal basis $\{\ket{+_{\lambda,\alpha}, \ket{-_\lambda,\alpha}}\}$ as described in \ref{meas}.
		
		iv. Corrections -- conditionally applying $X$ or $Z$ gates onto qubits $i \in V$.
	\end{definition}
	The \emph{graph} (state) for a measurement pattern is determined by the set of tuples $(i,j) \in V \times V$ such that we apply $CZ_{ij}$. Along with specifying measurement axes and angles for non-outputs $\bar{O} := V\I$, we get a labelled open graph.
	
	\begin{definition}
		A \bd{labelled open graph} is a tuple $\Gamma = (G,I,O,\lambda)$ where $G = (V,E)$ is a graph, $I,O \subseteq V$ are (input and output) subsets of $V$ and $\lambda : \bar{O} \to \{XY,YZ,XZ\} \times [0,2\pi)$.
	\end{definition}
	Every measurjement yields an outcome $0$ or $1$. In total there are $2^{|\bar{O}|}$ possible outcomes -- these are called \emph{branches} of the measurement pattern.
	
	After corrections we can sometimes ensure that every branch yields the same output. When this is possible, the measurement pattern is called \bd{deterministic}. This is formalised via \emph{flow}.
	
	The \bd{linear map} associated with a deterministic MBQC then looks like,
	\begin{equation}
	\underbrace{\left(\prod_{i \in \bar{O} } \bra{+_{\lambda(i),\alpha(i)}}\right)}_{\text{(det) measurement }}\underbrace{\left(\prod_{i \sim j} CZ_{ij}\right)}_{\text{entangling}}	\underbrace{\left( \prod_{i \in \bar{I}} \ket{+}_i\right)}_{\text{preperation}}
	\end{equation}
	This acts on the input state $\bigotimes_{i \in I} \ket{\psi_i}$. The \emph{resource state} used is,
	\begin{equation}
		\left(\prod_{i \sim j} CZ_{ij}\right) \left(\prod_{i \in \bar{I}} \ket{+}_i\right) \ket{\psi}_I
	\end{equation}
	
	
	
	\subsection{Corrections}
	Note that we cannot simply apply corrections by applying a $Z$ or $X$ gate on an incorrect outcome (typically a hardware/cost limitation on MBQC). Let us construct the stabiliser of the graph state.
	
	First note that $\ket{+}_u = X_u \ket{+}_u$ and the identity,
	\begin{equation}
		CZ_{uv} X_u = X_u Z_v CZ_{uv}
	\end{equation} 
	Then, for any $w \in \bar{I}$,
	\begin{align}
		\left(\prod_{u \sim v} CZ_{u,v}\right) \left(\prod_{u \in \bar{I}} \ket{+}_u\right) &= \left(\prod_{u \sim v} CZ_{u,v}\right) X_w \left(\prod_{u \in \bar{I}} \ket{+}_u\right)\\ &= \left(\prod_{v \in N_G(w)} Z_v \right) X_w \left(\prod_{u \sim v} CZ_{u,v}\right)
	\end{align}
	where we use the identity above when $CZ$ involves vertex $w$, and otherwise it commutes with $X_w$. Also, the $Z$ string commutes with $X_w$. 
	
	So, we see that,
	\begin{equation}
		 \left[ \left(\prod_{v \in N_G(w)} Z_v \right) X_w  \right]	  \left(\prod_{u \sim v} \ket{+}_u\right) = \left(\prod_{u \sim v}	 \ket{+}_u\right)
	\end{equation}
	For each vertex we get a stabiliser string of $Z$ on the neighbours and $X$ on the vertex.
	
	 Equivalently, we obtain the following operator equivalence that will allow us to perform corrections,
	 \begin{equation}
	 	X_w \left(\prod_{u \sim v} \ket{+}_u \right)=  \left(\prod_{v \in N_G(w)} Z_v \right) \left(\prod_{u \sim v} \ket{+}_u \right)
	 \end{equation} 
	The above case is easy -- to correct by applying a local $X$ on vertex $w$ can be achieved by applying a $Z$ to all neighbours of $w$. 
	
	We can also take the product of stabilisers indexed (defined) by many vertices. To corrct by applying a local $Z$, we can take the stabiliser of vertices in in its neighbourhood.
	
	Generally, we look for a stabiliser of the graph state,
	\begin{equation}
		S = Q_v \otimes P_{rest}
	\end{equation}
	where $Q_v$ is the local correction operator on vertex $v$. This is dependent (only) on the choice of plane and angle. The string of paulis $P_{rest}$ is called an \emph{extraction string}. Simple algebraic manipulation of a chosen stabiliser is the generator for corrections.
	
	We want $P_{rest}$ to be supported on the future (unmeasured) qubits. This is possible when we have a \emph{flow} condition on the graph.
	
	\subsection{Flow} The simplest case is causal flow, we assume that all vertices are measured in the $XY$ basis.
	\begin{definition} \bd{(causal flow).}  \\
		Given a labelled open graph $\Gamma = (G,I,O,\lambda)$ such that $\lambda(v) = XY$ for all $v \in \bar{O}$, a causal flow is a tuple $(f, \prec)$ where $f: \bar{O} \to \bar{I}$ and $\prec$ is strict partial order on $V$ satisfying,
		
		i. $v \sim f(v)$
		
		ii. $v \prec f(v)$
		
		iii. $\forall u \in N_G(f(v))$, $u=v$ or $v \prec u$
	\end{definition}
	The partial ordering $\prec$ gives us the order in which to perform measurements. 
	
	Here, an error on any $v \in \bar{O}$ can be corrected via applying $Z_v$. Consider the stabiliser of vertex $f(v)$, rearranging we get,
	\begin{equation}
		\prod_{w \in N_G(f(v) \setminus \{v\})} \hspace{-0.7cm} Z_w X_{f(v)} = Z_v
	\end{equation}
	Allowing measurements in all three planes and products of vertex stabilisers gives us generalised flow. 
	
	The following algebraic relation is very useful, let the stabiliser be specified by a set $g(v) \subseteq \bar{I}$,
	\begin{align}
		\prod_{u \in g(v)} \left(\prod_{w \in N_G(u)} Z_w\right) X_u &= \left(\prod_{u \in g(v)} \prod_{w \in N_G(u)} Z_w \right) \left(\prod_{u \in g(v)} X_u\right) \\ &= \left(\prod_{u \in \mathrm{odd(g(v))}} Z_u \right)\left(\prod_{u \in g(v)} X_u\right) \\ &= \left(\prod_{u \in \mathrm{odd}(g(v)) \setminus g(v)} Z_u\right) \left(\prod_{u \in g(v) \cap \mathrm{odd}(g(v))} Y_u\right) \left(\prod_{u \in g(v) \setminus \mathrm{odd}(g(v))} X_u\right)
	\end{align}
	where, in the second line we use that $Z_w^2 = I$, and only $Z$ on the odd neighbourhood of $g(v)$ survive.
	
	We then write the expression as a disjoint product -- either,
	
	i. $u \in \mathrm{odd}(g(v))$ and $u \notin g(v)$ -- apply $Z$
	
	ii. $u \in g(v)$ and $u \in \mathrm{odd}(g(v))$ -- apply $Y$
	
	iii. $u \in g(v)$ and $u \notin \mathrm{odd}(g(v))$ -- apply $X$
	
	Finally, based on whether we require a $X_v, Y_v$ or $Z_v$ correction, we choose $g(v)$ appropriately such that $v$ belongs to one of the corresponding sets. We search for subsets $g(v)$ and the ordering $\prec$ simultaneously, not separately.
	
	\begin{definition} \bd{(generalised flow).} \\
		Given a labelled open graph $\Gamma = (G,I,O,\lambda)$ such that $\lambda(v) \in \{ XY, XZ, YZ\}$ for all $v \in \bar{O}$, a \emph{generalised flow} or \emph{glow} for $\Gamma$ is a tuple $(g,\prec)$ where $g: \bar{O} \to \mathcal{P}(\bar{I})$ and $\prec$ is a strict partial order over $V$ satisfying for all $v \in \bar{O}$,
		
		i. for all $u \in g(v)$, $v \neq u \implies v \prec u$
		
		ii. for all $u \in \mathrm{odd}(g(v))$, $v \neq u \implies v \prec u$
		
		iii. $\lambda(v) = XY \implies v \in \mathrm{odd}(g(v)) \setminus g(v)$
		
		iv. $\lambda(v) = XZ \implies v \in g(v) \cap \mathrm{odd}(g(v))$
		
		v. $\lambda(v) = YZ \implies v \in g(v) \setminus \mathrm{odd}(g(v))$ 
	\end{definition}
	Conditions i, ii ensure that correction sets are in the \emph{future}, and iii - v follow from the stabiliser algebra.

We can generalise this by noting that if some measurements are pauli,  $\lambda(u) \in \{X,Y,Z\}$, then $u$ can be in the correcting set of some $v$ in the future $(u \prec v)$ as long as correcting $v$ induces the same pauli $\lambda(u)$ on $u$. This is described as a \emph{free correction in the past}. Note that the $\pm$ sign must still be tracked, but doesn't require tha application of any operator. 

\begin{definition}
\bd{(Pauli flow).} \\
Given a labelled open graph $\Gamma = (G,I,O,\lambda)$, a \emph{Pauli flow} for $\Gamma$ is a tuple $(p,\prec)$ where $p: \bar{O} \to \mathcal{P}(\bar{I})$ and $\prec$ is a strict partial order over $V$ satisfying for all $v \in \bar{O}$,

 $\left[\prec.X \right]$ for all $u \in p(v)$, $v \neq u$ and $\lambda(u) \notin \{X,Y\} \implies v \prec u$.
 
$\left[\prec.Z \right]$ for all $u \in \mathrm{odd}(p(v))$, $v \neq u$ and $\lambda(u) \notin \{Y,Z\} \implies v \prec u$.
 
 $\left[\prec.Y \right]$ for all $v \nprec u$, $v \neq u$ and $\lambda(u) = Y \implies u \notin p(v) \triangle \mathrm{odd}(p(v))$
 
 $\left[\lambda.XY \right]$ $\lambda(u) = XY \implies v \in \mathrm{odd}(p(v))$ and $v \notin p(v)$ 
 
 $\left[\lambda.XZ \right]$ $\lambda(u) = XZ \implies v \in p(v)$ and $v \in \mathrm{odd}(p(v))$ 
  
   $\left[\lambda.YZ \right]$ $\lambda(u) = YZ \implies v \in \mathrm{odd}(p(v))$ and $v \notin p(v)$ 
   
    $\left[\lambda.X \right]$ $\lambda(u) = X \implies v \in \mathrm{odd}(p(v))$ 
    
     $\left[\lambda.Z \right]$ $\lambda(u) = Z \implies v \in p(v)$ 
     
      $\left[\lambda.Y \right]$ $\lambda(u) = Y \implies v \in p(v) \triangle \mathrm{odd}(p(v))$
      
      where, $\triangle$ denotes the symmetric difference of sets. 
\end{definition}

The first three conditions are on the ordering $\prec$ similar to gflow, with the additional exceptional case for planar measurements in the past. The other six conditions are identical to gflow, with the last three explicitly written for the planar case.

We can describe the stabiliser algebra for the correcting set $p(v)$ as,
\begin{equation}
	S(p(v)) = \prod_{w \in p(v)} X_w \prod_{w \in \mathrm{odd}(p(v))} Z_w
\end{equation}
The local operator at some vertex $u \in p(v)$ is,	
\begin{equation}
	P_u = X^{1_{u \in p(v)}} Z^{1_{u \in \mathrm{odd}(p(v))}}
\end{equation}
where, $1_{x \in A}$ denotes the characteristics function -- 1 if $x \in A$ and 0 if $x \notin A$. 

The third axiom [$\prec.Y$] describes the $(0,0)$ corresponding to  a local $I$ and $(1,1)$ corresponding to a local $Z$ cases. Note that writing $u \notin p(v) \cap \mathrm{odd}(p(v))$ would not allow the trivial identity case.
	
	\newpage
	\section{References }
	
	\begin{enumerate}
		\item Pauli and Stabilisers -- \href{https://cs.uwaterloo.ca/~watrous/TQI/TQI.pdf}{Watrous TQI} \label{TQI}\\
		\item MBQC -- \label{MBQC} \\
		\item PDDAG --
	\end{enumerate}
	
	
	
	\section{Pauli Flow}
	
	
	\section{PDDAG}
	
	
	
	
\end{document}
